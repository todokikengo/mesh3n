\documentclass[12pt]{optlab-bachelor}
\usepackage{amsfonts}
\usepackage[T1]{fontenc}
\usepackage{amsmath,amssymb}
\usepackage{algorithm}
\usepackage{algorithmic}
\usepackage[dvipdfmx]{graphicx}

\def\年度{2018}
\def\氏名{十時健伍}
\def\学生番号{15715045}
\def\題目{正方格子グラフにおける信頼度評価問題}
\def\背題目{題目}

\renewcommand{\bibname}{参考文献}
\newcommand{\argmin}{\mathop{\rm arg~min}\limits}
\begin{document}
\frontmatter
%%%%%%%%%%%%%%%%%%%%%%%%%%%%%%%%%%%%%
\chapter{はじめに}
\section{研究背景}
現在,世の中では様々なネットワークが存在する.コンピューターや回線を
結ぶ回線ネットワークや交通ネットワークがその一例である.
それらのネットワークは要素の一部が故障する可能性がある.交通ネットワークの場合,
人身事故による路線封鎖や,交通整備による道路封鎖などが考えられる
これらの問題をモデル化する為に,辺の故障を考慮した$link$ $network$や辺と頂点の故障を
考慮した$ordinary$ $network$が考えられている.
ネットワークが稼働している確率をそのネットワークの信頼度と言う.信頼度の中でも全ての頂点が連結
する場合を全点間信頼度と呼び,この論文では全点間信頼度を扱う.
従来研究では,$link$ $network$の場合,どのようなグラフでもファクタリング理論と二分決定図に基づいた
アルゴリズムが効率的な信頼度評価方法として証明されている.
しかし,辺集合と頂点集合に特徴がある正方格子グラフの場合,信頼度を求める際には
ファクタリング理論や二分決定図より効率的なアルゴリズムがあると予測される

アイウエオ\cite{a-GAS}.

\chapter{諸定義}
\section{研究背景}
\subsection{ネットワーク}
\begin{itemize}
  \item aaaaaaa
\end{itemize}

\begin{description}
  \item[Step 1-1 : ] aaaaaaa
\end{description}

\begin{definition}
\end{definition}

\begin{theorem}
\end{theorem}
\begin{itemize}
  \item ファクタリング理論
\end{itemize}
ファクタリング理論を使うことで,グラフ$G_L$は以下の式で再帰的に表現される
$$R(G_{L(k)})=P(e_i)*R(G_{L(k)} \bullet e_i)+(1-P(e_i))*R(G_{L(k)} - e_i)$$
\begin{lemma}
\end{lemma}

\begin{proof}
\end{proof}

\begin{observation}
\end{observation}

\begin{proposition}
\end{proposition}

ああああああああああ
\chapter{成果}
\chapter*{謝辞}

%%%%%%%%%%%%%%%%%%%%%%%%%%%%%%%%%%%%%
\bibliographystyle{splncs03}
\bibliography{thesis}
\begin{flushright}
  2018年1月31日 \氏名
\end{flushright}
\endmatter
\end{document}
